% Created 2017-10-23 Mon 00:20
\documentclass[11pt]{article}
\usepackage[utf8]{inputenc}
\usepackage[T1]{fontenc}
\usepackage{fixltx2e}
\usepackage{graphicx}
\usepackage{longtable}
\usepackage{float}
\usepackage{wrapfig}
\usepackage{rotating}
\usepackage[normalem]{ulem}
\usepackage{amsmath}
\usepackage{textcomp}
\usepackage{marvosym}
\usepackage{wasysym}
\usepackage{amssymb}
\usepackage{hyperref}
\tolerance=1000
\author{Christopher Cundy}
\date{\today}
\title{to\_do}
\hypersetup{
  pdfkeywords={},
  pdfsubject={},
  pdfcreator={Emacs 25.1.1 (Org mode 8.2.10)}}
\begin{document}

\maketitle
\tableofcontents

\section{To do for paper}
\label{sec-1}
\subsection{Edit all}
\label{sec-1-1}
\subsection{Look up templates etc, double-check there's not one out there}
\label{sec-1-2}
\subsection{Do and write up the experiments}
\label{sec-1-3}
\subsubsection{Synthetic Task}
\label{sec-1-3-1}
\begin{enumerate}
\item Already done, written up. Need to review
\label{sec-1-3-1-1}
\end{enumerate}
\subsubsection{Medical Task}
\label{sec-1-3-2}
\begin{enumerate}
\item Being done, need to check on testing set
\label{sec-1-3-2-1}
\item Need to write up
\label{sec-1-3-2-2}
\end{enumerate}
\subsubsection{Throughput Task}
\label{sec-1-3-3}
\begin{enumerate}
\item Chat to Eric about this
\label{sec-1-3-3-1}
\item Just generate random data
\label{sec-1-3-3-2}
\end{enumerate}
\subsubsection{PLR/SLR comparison}
\label{sec-1-3-4}
\begin{enumerate}
\item Have the pure kernel data
\label{sec-1-3-4-1}

\item Do that for random data
\label{sec-1-3-4-2}
\item Do a 2-layer network, with varying seq$_{\text{len}}$
\label{sec-1-3-4-3}
\item Report the throughput
\label{sec-1-3-4-4}

\begin{center}
\begin{tabular}{lrrrr}
Sequence Length & LS-LSTM & SRU & QRNN(2) & QRNN(10\\
\hline
16 & 0.61 & 0.28 & 0.38 & 0.78\\
256 & 0.91 & 0.84 & 0.86 & 0.99\\
4,096 & 0.98 & 1.38 & 1.18 & 1.05\\
65,536 & 1.41 & 9.21 & 6.68 & 2.05\\
\end{tabular}
\end{center}
For an input$_{\text{size}}$ of 24, output 2, hidden$_{\text{size}}$ 256, 
batch$_{\text{size}}$ 65536 / seq$_{\text{len}}$
\end{enumerate}

\subsection{Collate the experiments and smooth}
\label{sec-1-4}
\subsection{Think about the overall message of the piece and rewrite as necessary}
\label{sec-1-5}
\subsection{Send to Adam, Andy to check}
\label{sec-1-6}
\subsection{Submit}
\label{sec-1-7}
\subsubsection{Then go to party and relax the fuck out}
\label{sec-1-7-1}


\section{Timeline}
\label{sec-2}
\subsection{Saturday evening:}
\label{sec-2-1}
\subsubsection{Done interesting medical results, tested on testing set}
\label{sec-2-1-1}
\subsubsection{Looked at throughput task}
\label{sec-2-1-2}
\subsubsection{Done PLR / SLR throughput task}
\label{sec-2-1-3}
\subsection{Sunday evening:}
\label{sec-2-2}
\subsubsection{Fully written up medical task}
\label{sec-2-2-1}
\subsubsection{Written up throuput and PLR/SLR task}
\label{sec-2-2-2}
\subsubsection{Start editing the paper, smoothing into a cohesive whole}
\label{sec-2-2-3}
\subsection{Monday evening:}
\label{sec-2-3}
\subsubsection{Finish editing, send off to everyone else}
\label{sec-2-3-1}


\section{Paper structure}
\label{sec-3}
\subsection{Introduction, abstract, background}
\label{sec-3-1}
\subsubsection{Describe}
\label{sec-3-1-1}
We can apply the PLR method to any architecture that satisfies the 
constraints. E.g. SRU, QRNN, etc. Introduce LS-LSTM as good substitute.
Describe how previous papers have shown that linear LSTMs work well, this
approach allows us to speed it up.
\subsection{Experiments}
\label{sec-3-2}
\subsubsection{Benchmarks}
\label{sec-3-2-1}
\begin{enumerate}
\item Throughput of pure PLR kernel vs SLR kernel
\label{sec-3-2-1-1}
\item Show how it speeds up the SRU, QRNN, and LS-LSTM
\label{sec-3-2-1-2}
\item Show how the LS-LSTM has much better throughput than the CudnnLSTM
\label{sec-3-2-1-3}
\end{enumerate}
\subsubsection{Synthetic task to show that linear LSTMs can still work well}
\label{sec-3-2-2}
\subsubsection{Medical task}
\label{sec-3-2-3}



31.4 * (4.3 * 6) = 
\textasciitilde{}50 * \textasciitilde{}74
135 * 190
% Emacs 25.1.1 (Org mode 8.2.10)
\end{document}